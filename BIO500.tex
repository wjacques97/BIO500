\documentclass{article}
\usepackage[utf8]{inputenc}

\title{Effets du covid}
\author{wil50000 }
\date{April 2021}

\begin{document}

\maketitle

\section{Introduction}
Introduction :  
En ces temps poil pandémiques et maussades, l’ensemble de la population est affecté par de drastiques changements de mode de vie nécessaires pour diminuer la progression du virus et de ses variants. Par exemple, on observe une dégradation majeure de la santé mentale chez les étudiants depuis le début de la prise de mesure lié à la Covid-19 (Son et al., 2020). En effet, toujours selon cette source, on note une augmentation du stress, de l’anxiété et des pensées dépressives. Toutefois, depuis le confinement, il semble que leur performance académique ce soit améliorée puisque le nombre de distraction a diminué, ce qui leur a permis de développer des méthodes d‘étude plus efficaces (Gonzalez et al., 2020). Une chose est certaine c’est que les mesures liées à la Covid-19 ont un puissant effet sur les étudiants. Nous nous sommes d’ailleurs intéressés à l’impact de ces mesures sur la dynamique de la formation des équipes dans le cadre de travaux scolaires. Plus précisément, nous nous intéresserons à l’effet des mesures lié à la Covid-19 sur le nombre de cours en lignes et le nombre de lien différents entre les étudiants. Selon nous, il devrait y avoir davantage de cours en ligne et la diversité des équipes devrait diminuer. De plus, nous analyserons l’hypothèse que les étudiants sont plus conservateurs dans leur choix de coéquipier, pendant la période post Covid-19 ainsi que dans les cours en ligne comparativement à la période pré Covid-19 et dans les cours en présentiel. 
\end{document}
